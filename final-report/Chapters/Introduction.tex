\clearpage
\chapter{Introduction}

Vehicles are one of the most important means of road transport for goods and services. Almost all the countries around the world have complicated road networks connecting remote locations. These road networks facilitate easy transport of people, goods, and services further encouraging the development of more networks. There has been exponential growth in the automobile industry to provide and market various automobiles into these road networks. This requires the need of policing this sector to ensure the safety of the people and the country. Controlling and managing this huge network is a great challenge due to its very vast and complicated network supporting a large population of vehicles. As part of policing, cameras are set up at various locations to monitor the traffic. However there exist the tedious work of manually analyzing the camera feed to detect and control the traffic. This project aims to tackle this particular issue.

\section{Problem Statement}
A novel approach to provide an integrated platform utilizing existing infrastructure to monitor transport sector, and extend timely information to policing agencies, with intuitive interface.

\section{Background}
Road traffic is one of the dynamic public sectors that involve huge human interventions. These include, but are not limited to construction of roads, transport of goods and other products, general public using public and private travel means, movement of construction goods, etc. The automobile is a booming industry, with a lot of companies producing new automobiles for the market meeting the need of the public. These include construction vehicles such as JCB, tractors, excavators, road rollers, trucks, etc; public goods vehicles such as vans, trucks, pickups; public vehicles such as buses, autos, taxis, etc; private vehicles such as cars, bikes, etc. Manufactures continuously supply new and modern automobiles to support the day-to-day activities of the public, and hence the nation.

This huge flow of vehicles demanded the setting up of rules and regulations for the safety of the people and properties. These rules ensured a smooth flow of traffic. However, at times these rules are broken by the public. To re-enforce these regulations proper policing is required. Officials are deployed at various points to physically inspect and regulate traffic. Upon finding any violation, the offender is fined. With the involvement of technology, speed cameras are set up on major roads to ensure that all vehicles do not travel beyond the safe speed limit. These cameras are special cameras that are triggered only upon finding a vehicle moving at high speed. When triggered they capture the license plate of the vehicle along with the time and place of the event. Obtaining license plate info helps to fetch associated vehicle information such as owner, address, vehicle make, model, type, etc. However, these are special cameras and a too costly to be deployed in larger and remote places.

At times, it was necessary to re-trace the route followed by a particular vehicle. As the technology grew, surveillance cameras are set up at various locations. Cameras allowed continuous monitoring of a particular section of the road. The feeds are saved at the severs and officials can replay the tape to extract the necessary information. These cameras are far less expensive than the speed cam, but also have lower resolution. These cameras find it difficult to extract unique features such as license plate information. They are mainly used for surveillance purposes with the just aim to record the event occurring in a fixed sector.  

As the camera network was set up, the process of policing is still a tedious task. Officials need to manually replay each camera feed. Most of the time, the feed would be empty showing there was no traffic at that moment. Moreover, there would be many types of vehicles to consider. It is also possible for the vehicles to take alternative routes. The manual process of monitoring each feed is a highly time-consuming and tired-sum process. Moreover, it may need to identify a particular vehicle before it travels out of a particular zone.

This calls for the need for a system that could traverse the camera network detecting useful video frames and analyzing them to extract relevant information. This information can include features like type, color, model, location, time of the event, etc. A query can be issued to look up vehicles matching such descriptions. This saves a lot of time by skipping the dead frames. Another advantage is that, during an investigation, the witness can only provide such features to testimony an event. The suspect may be able to forge the license plate, but not the whole vehicle description. 

\section{Features}
\begin{itemize}
	\item \textbf{Traffic Analysis and Vehicle Tracking:} AI based traffic analysis, vehicle re-identification and tracking over the existing infrastructure. 
	\item \textbf{Intuitive search:} A minimalistic and simple searching that even Naive users can utilize with ease, to trace vehicles.   
	\item \textbf{Integration of any IP camera:} IP camera accessible through internet or local LAN or over VPN, can be integrated with ease.
	\item \textbf{Secured IP streaming:} Users can securely stream authorized IP camera without needing to access Camera Infrastructure Network using HLS over HTTPS.
	\item \textbf{Granular Role based permission management:} Enables controlled management of Camera Infrastructure with each fine-tuned permission management.
	\item  \textbf{Map based CCTV visualization:} All IP cameras are configured with GPS coordinates, and other details, allowing easy visualization in Maps. 
\end{itemize}


\section{NVIDIA AI City Challenge}
This project is one of the open challenges in NVIDIA AI city challenge, to perform Natural language-based vehicle track retrieval and City-scale multi-camera vehicle tracking. Various research and study are being conducted on this field.

\subsection*{Definition}
AI City means applying AI to improve the efficiency of operations in city environments. This manifests itself in improving transportation outcomes by making traffic more efficient and making roads safer, improving building operations by making them more energy efficient, reducing friction in retail environments by speeding up traffic at retail checkout, etc. The common thread in all these diverse uses of AI is the extraction of actionable insights from a plethora of sensors through real-time streaming and batch analytics of the vast volume and flow of sensor data, such as those from cameras.

\subsection*{The AI City Challenge Workshop 2022}
The AI City Challenge Workshop at CVPR 2022 specifically focus on problems in two domains where there is tremendous unlocked potential at the intersection of computer vision and artificial intelligence
\begin{enumerate}
	\item The Intelligent Traffic Systems (ITS)
	\begin{itemize}
		\item City-scale multi-camera vehicle tracking
		\item Natural language-based vehicle track retrieval
		\item Naturalistic driver data analytics
		\item Anomaly Detection
	\end{itemize}
	\item The Brick and Motar retail business
	\begin{itemize}
		\item Automated checkout
		\item Efficient Store utilization
	\end{itemize}
\end{enumerate}

\section{Application}
Mere processing of footage from camera have wide range of applications, especially in an dynamic sector like road transport. Due to urbanization and expansion of road networks, supported by the growth of technology and digitization, applications can vary from monitoring, planning to research, security and decision making. Some of the possible application of analyzing the traffic using camera network are:
\begin{itemize}
	\item Identify and track the route taken by a targeted vehicle.
	\item Identify violations such as the check for helmet and passenger count in two wheeler, illegal turn taken at one-way roads, non-permitted vehicle entering the perimeter (eg: heavy trucks via old bridge is prohibited) etc.
	\item Identify events such as accidents or unexpected crowding, etc.
	\item Conduct research on the flow of traffic aiding for decision making for new road construction.
	\item Provide decision parameter for efficient routing of traffic to remove traffic blocks. Helps to change traffic load for emergency services such as ambulance, fire-trucks etc.
	\item Can be used to checkpoint vehicles, enhancing integrity and security of services. For example:
	\begin{itemize}
		\item Confirming that cargo have left the station and is in-route to its destination.
		\item Track and predict public transport services.
	\end{itemize}
\end{itemize}
