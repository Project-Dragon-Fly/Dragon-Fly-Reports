\chapter{Methodology}
\lipsum[1-2]


\section{Training}
\lipsum[1]

\subsection{Method Overview}
\lipsum[1-2]

The $P$ is defined as the following:
\begin{equation}
	P(\tau | \tau_{ref},C) = \frac{1}{Z} exp(-c(\tau, \tau_{ref},C)) \\
\end{equation}
Where $	Z = \int_{\tau} P(\tau | \tau_{ref},C)$ is the normalization factor and $c(\tau, \tau_{ref},C) \in \R_+$ is a cost function indication proximity to the reference and distance from obstacles. The trajectory cost function is defined as 
\begin{equation}
	c(\tau, \tau_{ref},C) = \int_0^1 \lambda_c C_{collision}(\tau(t)) + \int_0^1 [\tau(t)-\tau_{ref}(t)]^T Q[\tau(t)-\tau_{ref}(t)]{dt}
	\label{eqn:collision}
\end{equation}
where $\lambda_c = 1000$, $Q$ is a positive semi-definite state cost matrix and $C_{collision}$ is a measure of the distance of the quadrotor to the points in $C$ 

\lipsum[1-2]